
\documentclass[%singlesided,
               doublesided,
               paper=a4,
               fontsize=10pt
              ]{resume}
\usepackage{hyperref}
\usepackage{emoji}
\usepackage{xcolor}

%%%%%%%%%%%%%%%%%%%%%%%%%%%%%%%%%%%%%%%%%%%%%%%%%%%%%%%%%%%%%%%%%%%%%%%%%%%%%%%%
% set geometry
%%%%%%%%%%%%%%%%%%%%%%%%%%%%%%%%%%%%%%%%%%%%%%%%%%%%%%%%%%%%%%%%%%%%%%%%%%%%%%%%

\setlength\highlightwidth{8cm}
% note that margintop gets added to this value, i.e. the header bar is 5cm
\setlength\headerheight{4cm}
\setlength\marginleft{1cm}
% needs to be 1.5 times to be actually equal. why?
\setlength\marginright{\marginleft}
\setlength\margintop{1cm}
\setlength\marginbottom{1cm}


%%%%%%%%%%%%%%%%%%%%%%%%%%%%%%%%%%%%%%%%%%%%%%%%%%%%%%%%%%%%%%%%%%%%%%%%%%%%%%%%
% FONTS
%%%%%%%%%%%%%%%%%%%%%%%%%%%%%%%%%%%%%%%%%%%%%%%%%%%%%%%%%%%%%%%%%%%%%%%%%%%%%%%%

\RequirePackage{fontspec}
\setmainfont{Carlito}


%%%%%%%%%%%%%%%%%%%%%%%%%%%%%%%%%%%%%%%%%%%%%%%%%%%%%%%%%%%%%%%%%%%%%%%%%%%%%%%%
% COLORS
%%%%%%%%%%%%%%%%%%%%%%%%%%%%%%%%%%%%%%%%%%%%%%%%%%%%%%%%%%%%%%%%%%%%%%%%%%%%%%%%

\colorlet{highlightbarcolor}{lightgray}
\colorlet{headerbarcolor}{darkgray}

\colorlet{headerfontcolor}{white}
\colorlet{accent}{awesome-red}
\colorlet{heading}{black}
\colorlet{emphasis}{black}
\colorlet{body}{black}


%%%%%%%%%%%%%%%%%%%%%%%%%%%%%%%%%%%%%%%%%%%%%%%%%%%%%%%%%%%%%%%%%%%%%%%%%%%%%%%%
% set document
%%%%%%%%%%%%%%%%%%%%%%%%%%%%%%%%%%%%%%%%%%%%%%%%%%%%%%%%%%%%%%%%%%%%%%%%%%%%%%%%


\begin{document}

\name{Dr. Po-Hsun Tseng}

\tagline{6+ years experience of large-scale software development
         with \textbf{C programming} on the \textbf{Linux} system
         throughout my Ph.D. journey.
         Seeking a challenging and rewarding opportunity with
         low-level system programming (eg. device driver, Linux kernel, CPU architecture)
         to apply my skills at a fabless company.}

% make photo exactly match the header with margintop/marginright/marginbottom as margin
\photo[round]{headshot.jpg}{\dimexpr \headerheight-\marginbottom}

\makeheader

\highlightbar{

    \section{Contact}
    \email{zengbs@gmail.com}
    \phone{+886 966 587 832}
    \location{Hsinchu, Taiwan}
    \vspace{0.5em}
    %\homepage{nicokrieger.com}{https://www.nicokrieger.de}
    \github{https://github.com/zengbs}{https://github.com/zengbs}
    %\linkedin{Nico Krieger}{https://www.linkedin.com/in/nico-krieger-6b28151b2/}
    %\orcid{0000-0002-1868-0660}{https://orcid.org/0000-0002-1868-0660}
    %\ads{NASA/ADS publication list}{https://}}

    \section{Skills}

    \skillsection{Programming}
    \skill{C}{5}
    \skill{Bash scripting}{5}
    \skill{System programming}{3}
    \skill{CUDA}{4}
    \skill{Python}{5}

    \vspace{0.5em}
    \skillsection{Operating System}
    \skill{Linux -- user}{5}
    \skill{Linux -- kernel}{2}

    \vspace{0.5em}
    \skillsection{Architecture}
    \skill{ARM architecture}{2}

    \vspace{0.5em}
    \skillsection{Software \& Tools}
    \skill{Git}{5}
    \skill{Vim}{5}
    \skill{Gdb}{5}
    \skill{Valgrind}{5}
    %\skill{Visualisation}{5}
    %(matplotlib, gnuplot)\\
    %\skill{Data handling/analysis}{4}
    %(hdf5, numpy, scipy)


    \vspace{0.5em}
    \skillsection{Languages}
    \skill{Chinese (native)}{5}
    \skill{English -- writing}{4}
    \skill{English -- speaking}{3}
    \skill{English -- listening}{4}
    \skill{English -- reading}{5}
    \bigskip

    %\section{Certificates}
    %\simpleskill{AWS certified cloud practitioner}
    %\simpleskill{AWS certified ML Specialist}
    %\simpleskill{Databricks Lakehouse Platform}

}
\mainbar{
    %\section{About this template}
    %Section are set in bold face.
    %An optional parameter of \texttt{\textbackslash section} takes a symbol to
    %add in front of the text.
    %This option is used in the jobs and education sections below.


    \section[\faMortarBoard]{Education}
    \job
        {Ph.D. in Computational Physics}
        {National Taiwan University, Taiwan}
        {08/2016 - 06/2022}
        {\begin{itemize}
         \item Developed and implemented a new algorithm to
               reduce numerical error by $10^6$ compared to conventional one.\\
               See Fig. 16 in (\textcolor{violet}{\url{https://arxiv.org/abs/2012.11130}}).
         \item Designed a new approach to further promote the robustness of GAMER.
               The new approach was adopted in the research project led
               by Dr. Kuo-Chuan Pan from Tsing Hua University.
               (\textcolor{violet}{\url{https://github.com/gamer-project/gamer/pull/60}})
         \item Worked on improving the GAMER collaborated with
               Dr. Tzihong Chiueh and Dr. Hsi-Yu Schive.
               See the Section \textit{References}.
         \item A main contributor of the GAMER.\\
               (\textcolor{violet}
               {\url{https://github.com/gamer-project/gamer/graphs/contributors}})
         \item Built NIS, NFS, and Linux cluster from scratch with colleagues
               and worked with Dr. Hsi-Yu Schive to bootstrap simulations for research.
         \end{itemize}}
    \section[\faGears]{Work history}
    \job{01/2015 - 02/2016}
        {TDK corporation, Singapore}
        {Circuit designer}
        {\begin{itemize}
           \item Designed the circiut of surface acoustic wave(SAW) filters
         \end{itemize}
        }
    \military{08/2013 - 08/2014}
        {Military service}

    \section[\faMortarBoard]{Education}
    \job
        {M.Sc. in Physics}
        {National Taiwan University, Taiwan}
        {09/2011 - 07/2013}{}
    \vspace{1.5em}
    \job
        {B.Sc. in Mathematics}
        {National Central University, Taiwan}
        {09/2006 - 07/2011}{}

    %\section{Achievement}%, honours and awards}
    %\achievement{My first achievement}

    \section{Soft skills}
    \tag{Nonverbal communication}
    \tag {Active listening}
    \tag{Open mindedness}
    \tag{Patience}
    \tag{Mutual respect}
    \tag{Teamwork}
    \tag{Brainstorming}
    \tag{Collaboration}

    \section{General Skills}
    %\smallskip % additional skip because tag outlines use up space
    \tag{Numerical algorithm}
    \tag{Large-scale project}
    \tag{cscope}
    \tag{makefile}
    \tag{GNU autotools}
    %\tag{to the next line}
    %\tag{if necessary}


    %\medskip
    %Tags must be ordered by hand with newlines to get a nice layout, especially for long tags.

    %\section{Wheel Chart}
    %% This is taken from AltaCV
    %% see https://github.com/liantze/AltaCV for details
    %\wheelchart{1.5cm}{0.5cm}{% outer and inner diameter
    %    6/8em/accent!20/Sleep,          % comma-separated list of
    %    8/8em/accent!40/Daytime job,    % fraction of 24 / line length / color / label
    %    2/8em/accent!80/Training,          % here, the color is shades of the accent color
    %    3/8em/accent!60/Recovering from fighting criminals,
    %    5/8em/accent/Being Batman
    %}
}
\makebody
\clearpage

%%%%%%%%%%%%%%%%%%%%%%%%%%%%%%%%%%%%%%%%%%
%% End of the first page
%%%%%%%%%%%%%%%%%%%%%%%%%%%%%%%%%%%%%%%%%%

\pagestyle{empty}

% The highlightbar needs to be filled to display mainbar contents correctly in singlesised mode
% For an empty highlightbar, fill with empty space
%\highlightbar{\hfill}
%\mainbar{
%    \section{Another section}
%    This page uses the page style \texttt{highlightmain}
%    which shows the highlight bar (gray) and the main part (white background)
%    but omits the header.
%    The default page style is \texttt{headerhighlightmain} with all three elements.
%    If you don't want header, nor highlight bar,
%    use page style \texttt{\textbackslash pagestyle\{empty\}}.
%    Neither main, nor highlight bar must be filled to make this template work.
%    It is possible to use a page style with the highlight bar
%    but leave it empty by setting an empty highlightbar \texttt{\textbackslash highlightbar\{\}}.
%
%    \vspace{0.5em}
%    \subsection{Subsection 1}
%    Demonstrate subsections.
%
%    \subsection{Subsection 2}
%    Subsection are also bold face but a smaller font then section. They also omit the rule.
%}
%\makebody
%
%\clearpage

%%%%%%%%%%%%%%%%%%%%%%%%%%%%%%%%%%%%%%%%%%
%% End of the second page
%%%%%%%%%%%%%%%%%%%%%%%%%%%%%%%%%%%%%%%%%%

\pagestyle{empty}

\section{Publications}
\pubforcefullwidth

% Demonstrate what an \texttt{\textbackslash pagestyle\{empty\}} page looks like.
% Also show off the macros for publications that uses small icons for authors,
% date, journal and links.
%
% Achieving a good looking spacing can be tricky.
% For empty pagestyles where the full width is available use
% \texttt{\textbackslash pubforcefullwidth} to force the publoication list
% to take up all the available space.
% The (relative) lengths reserved for date, journal and links can be set with the parameters
% \texttt{\textbackslash pubdatelength}, \texttt{\textbackslash pubjournallength} and
% \texttt{\textbackslash publinklength} as in
% \texttt{\textbackslash setlength\{\textbackslash pubdatelength\}\{0.15 \textbackslash linewidth\}}.
% \bigskip

\publication
	{An adaptive mesh, GPU-accelerated,
         and error minimized special relativistic hydrodynamics code} % Title
	{\textbf{Po-Hsun Tseng}, Hsi-Yu Schive, Tzihong Chiueh} % Authors
	{2021} % Year
	{Monthly Notices of the Royal Astronomical Society Vol. 504, pp. 3298-3315\\ % Journal
        \url{https://arxiv.org/abs/2012.11130}} % ADS & arxiv links
       {\ADS{https://ui.adsabs.harvard.edu/abs/2021MNRAS.504.3298T/abstract}
        } % ADS & arxiv links

\publication
	{The symmetry problem of the Fermi and eROSITA bubbles: A proof-of-concept study}
	{\textbf{Po-Hsun Tseng}, Hsiang-Yi Karen Yang, Hsi-Yu Schive,
         Chun-Yen Chen, Tzihong Chiueh} % Authors
	{2022} % Year
	{preprint}\\ % Journal




\section{Talks}
\begin{itemize}
\item
          An adaptive-mesh, GPU-accelerated, and optimally error-controlled
          special relativistic hydrodynamics code\\
          \begin{minipage}{6in}
       	  Oral (remote), American Center for Physics College Park, U.S.A
          \end{minipage}
          \hfill
          \begin{minipage}{1in}
          Mar. 2021
          \end{minipage}
\item
          A new and accurate code for simulating special relativistic hydrodynamics\\
          \begin{minipage}{6in}
          Oral, Annual Meeting of the Physical Society of Taiwan, NPTU.
          \end{minipage}
          \hfill
          \begin{minipage}{1in}
          Feb. 2020
          \end{minipage}
\end{itemize}
\section{References}
\begin{itemize}
  \item Please send an appointment letter to request a call. \emoji{smiley}
  \item Dr. Tzihong Chiueh\\
        Distinguished Professor, Institute of Astrophysics, National Taiwan University\\
        \location{Taipei 10617, Taiwan}
        \email{chiuehth@phys.ntu.edu.tw}
        \phone{+886 2 3366 8628}
  \item Dr. Hsi-Yu Schive\\
        Assistant Professor, Institute of Astrophysics, National Taiwan University\\
        \location{Taipei 10617, Taiwan}
        \email{hyschive@phys.ntu.edu.tw}
        \phone{+886 2 3366 8644}
  \item Dr. Hsiang-Yi Karen Yang\\
        Assistant Professor, Institute of Astronomy, National Tsing Hua University\\
        \location{Hsinchu 30013, Taiwan}
        \email{hyang@phys.nthu.edu.tw}
        \phone{+886 3 574 2953}
\end{itemize}


\end{document}
